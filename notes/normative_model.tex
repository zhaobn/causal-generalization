\documentclass[11pt]{article}

\usepackage[sort]{natbib}
\usepackage{fancyhdr}

% you may include other packages here (next line)
\usepackage{enumitem}
\usepackage[tiny]{titlesec}
\usepackage{lastpage}
\usepackage{amsthm}
\usepackage{amssymb}
\usepackage{amsmath}
\usepackage{verbatim}
%\usepackage{graphicx}
%\usepackage{wrapfig}

% \theoremstyle{definition}
% \newtheorem{definition}{Definition}[section]
% \newtheorem{theorem}{Theorem}
% \newtheorem{lemma}[theorem]{Lemma}

%\newtheorem{exmp}{Example}[section]


\newcommand{\fa}{f^{\alpha}}
\newcommand{\Fa}{F^{\alpha}}
\newcommand{\pre}{pre_{\sigma}}


%----- you must not change this -----------------
\oddsidemargin 0.2cm
\topmargin -1.0cm
\footskip 0cm
\textheight 24.0cm
\textwidth 15.25cm
\parindent=0pt
\parskip 1ex
\renewcommand{\baselinestretch}{1.1}
\pagestyle{fancy}
%----------------------------------------------------



% enter your details here----------------------------------

\lhead{\normalsize \textrm{Mysterious Stones}}
\chead{}
\rhead{\normalsize Bonan Zhao}
\lfoot{}
\cfoot{Page \thepage \hspace{1pt} of \pageref{LastPage}}
\rfoot{}
\setlength{\fboxrule}{4pt}\setlength{\fboxsep}{2ex}
\renewcommand{\headrulewidth}{0.4pt}
\renewcommand{\footrulewidth}{0.4pt}

	
\begin{document}


%----------------your title below -----------------------------

\begin{center}
{\bf Normative Model I}

{\normalsize \today}
\end{center}


%---------------- start of document body------------------

\section*{Language}

Let $O$ be the set of all stones (objects),
and use $A, R, R'$ to represent a stone's role in a causal interaction - 
	$A$ stands for the \textit{agent}, 
	$R$ for the pre-interaction state of the \textit{recipient}, 
	and $R'$ the post-interaction state of the recipient, also named the \textit{result}.

Feature space $F = \{ L, S, \ldots \}$ consists of
	\textit{lightness} $L$ - the color shadings,
	and \textit{sidedness} $S$ - number of edges of the polygons.
	Note that this feature set may grow as the experiment setting gets richer.

For this model, 
	lightness $L$ takes value $l_1, l_2, l_3, l_4$ along the lightness scale from light to dark. 
	Specifically, 
		$l_1$ stands for \textit{light}, 
		$l_2$ for \textit{medium}, 
		$l_3$ for \textit{dark}, and 
		$l_4$ for \textit{very dark}, as used in the experiment.
	Sidedness $S$ takes value $p_3, \ldots, p_7$, 
		where each subscript represents the number of edges for a polygon (hence $p$).

We define a value-reading function $v(o, f) = u, o \in O, f \in F$ and $u$ is the value of feature $F$ for object $o$.

For our current model, each stone has two features - lightness $L$ and sidedness $S$.
Hence for each object $o \in O$ there are two kinds of value-reading functions $v(o, L)$ and $v(o, S)$.
For simplicity, we will use $L(o)$ instead of $v(o, L)$ to read stone $o$'s lightness $L$,
	and $S(o)$ instead of $v(o, S)$ to read stone $o$'s sidedness $S$.



With these definitions, the language of this task consists of

\begin{itemize}
\item Atomic sentences:
	\begin{itemize}
		\item $L(A), L(R), L(R'), S(A), S(R), S(R')$: read stone feature values
		\item $1, \ldots, 7$: index numbers
		\item $l_1, \ldots, l_4, s_3, \ldots, s_7$: feature values
	\end{itemize}
\item Relations:
	\begin{itemize}
		\item $=, \neq$: compare if values match. Eg. 
					$L(A) = l_2, S(A) \neq S(R')$.
		\item $>, <$: compare values within the same feature by comparing their subscripts. 
					Eg. $l_1 < l_2, S(A) > S(R)$.
		\item $+, -$: plus and minus operations on feature value subscripts.
		 			Eg. $s_1 + 1 = s_2, L(R') = L(R) - 1$.
		% \item (To decide) min, max: get minimal/maximal value. Eg. min(${c_1, c_2}$) = $c_1$.
		\item $\models$: \textit{if} the cause conditions (the part before $\models$) satisfy, result effects (the part after $\models$) follow.
					Eg. $(c(R) = c(A)) \models (c(R') = c(R))$.
			If the cause conditions are not satisfied, we assume that the recipient stone does not change. 
			See below for legit candidates of a cause condition or result effect.
	\end{itemize}

\newpage
\item Grammar:
	\begin{itemize}
		\item If $\phi, \psi, \chi$ are an atomic sentences, the following compositions are \textit{basic sentences} of the language:
			$(\phi = \psi), (\phi \neq \psi), (\phi > \psi), (\phi < \psi)$,
			$(\phi + \psi = \chi), (\phi - \psi = \chi)$
		\item If $\alpha, \beta$ are basic sentences, $[\alpha, \beta]$ is the conjunction of $\alpha$ and $\beta$. 
			Number of conjuncts in such conjunctions $\geq 2$. 
		\item A causal hypothesis is of the form $[\alpha_1, \ldots, \alpha_k] \models [\alpha_{k + 1}, \ldots, \alpha_n]$, where
			\begin{itemize}
				\item Either the cause condition is $\top$ - \textit{any}, or each basic sentence of $\alpha_1, \ldots, \alpha_k$ contains $A$ or $R$, but not $R'$.
				\item Each basic sentence of $\alpha_{k + 1}, \ldots, \alpha_n$ contains $R'$.
			\end{itemize}
	\end{itemize}
\end{itemize}

Examples

\begin{itemize}
	\item $[(s(A) \neq s(R)), (c(A) \neq c(R))] \models [(c(R') = c(A)), (s(R') = s(A))]$ \\
	If the agent stone and the recipient stone have different shapes and colors, then the recipient stone will turn into the same as the agent stone;
	otherwise the agent will have no effect on the recipient stone.
	\item $(s(A) = s_3) \models (s(R') = s(R) + 1)$ \\
	If the agent stone is a triangle (regardless of its color), then the recipient stone shape's number of edges increases by 1;
	otherwise the agent will have no effect on the recipient stone.
\end{itemize}

\section*{Hypotheses}

Let's restrict the complete hypothesis space by only allowing meaningful sentences up to step 1. 

For $\alpha_{k + 1}, \ldots, \alpha_n$:
	let $\alpha_{k + 1}$ be $(s(R') = x)$, 
	where $x$ takes value from
	$s_3, \ldots, s_7, s(A), s(R), s(A) + 1, s(A) - 1, s(R) + 1, s(R) - 1$;
then, replace $=$ with $\neq, >, <$.
Apply the same procedure for $c(R')$ to compose $\alpha_{k+2}$. 
With 4 color shadings and 5 shapes this amounts to
$(4 + 6) \times 4 +  (5 + 6) \times 4 = 84$
effects.

As for the cause conditions part, i.e., $\alpha_1, \ldots, \alpha_k$, there are several options.

\begin{itemize}
 \item Restrain from using \textit{if}, the cause conditions part can be trivialized to just $\top$,
 	meaning that properties of the agent and recipient stones do not matter;
 	just being the agent or being the recipient suffices to produce the specified effect.
 \item Applying the procedure used for producing the effects to produce cause conditions. 
 	For example, let $\alpha_1$ be $(s(A) = x)$ where $x$ takes value from 
 	$s_3, \ldots, s_7, s(R), s(R) + 1, s(R) - 1$, 
 	then replace $=$ with $\neq, >, <$. 
 	Let $\alpha_2$ be $(c(A) = x)$ and apply the same procedure.
 	Similar for $(s(R) = x)$ and $(c(R) = x)$.
 	Note that $(s(A) = s(R))$ is equivalent to $(s(R) = s(A))$, 
 	therefore when generating cause conditions using the $(s(R) = x)$ form, duplicates must be removed.


\newpage
 \item A complex reasoner can combine multiple causal hypotheses instead maintaining only one - in a \textit{if ... else ...} manner.
\end{itemize}


\section*{Learning}

Let $\langle A, R, R' \rangle$ be a complete data point $d$. 
For a causal hypothesis $h$,
if $s(A), c(A), s(R), c(R)$ make the cause conditions true and 
with $s(R'), c(R')$ they make the result effects true, 
then $P(d|h) = 1$.
If $s(A), c(A), s(R), c(R)$ fails to satisfy the cause conditions,
recipients should remain as they are (because no causes are posed onto them),
hence $P(d|h) = 1$ if $s(R') = s(R)$ and $c(R') = c(R)$, 
and $P(d|h) = 0$ otherwise.

Thus, upon observing a complete data point $d$, 

\[P(h|d) = \frac{P(d|h)P(h)}{\sum_{h_i \in H}P(d|h_i)P(h_i)}\]

Assuming a flat prior for the first data point,
our Bayesian learner updates hypothesis space 6 times sequentially upon 
observing the six learning shots.

\section*{Generalization}

Upon observing a partial data point $d' = \langle A, R \rangle$,
the complete data point $d^*$ normalizes over 20 possible $R'$s.



















% ----------------end of document body---------------------



\end{document}