\documentclass[11pt]{article}

\usepackage[sort]{natbib}
\usepackage{fancyhdr}

% you may include other packages here (next line)
\usepackage{enumitem}
\usepackage[tiny]{titlesec}
\usepackage{lastpage}
\usepackage{amsthm}
\usepackage{amssymb}
\usepackage{amsmath}
\usepackage{verbatim}
%\usepackage{graphicx}
%\usepackage{wrapfig}

% \theoremstyle{definition}
% \newtheorem{definition}{Definition}[section]
% \newtheorem{theorem}{Theorem}
% \newtheorem{lemma}[theorem]{Lemma}

%\newtheorem{exmp}{Example}[section]


\newcommand{\fa}{f^{\alpha}}
\newcommand{\Fa}{F^{\alpha}}
\newcommand{\pre}{pre_{\sigma}}


%----- you must not change this -----------------
\oddsidemargin 0.2cm
\topmargin -1.0cm
\footskip 0cm
\textheight 24.0cm
\textwidth 15.25cm
\parindent=0pt
\parskip 1ex
\renewcommand{\baselinestretch}{1.1}
\pagestyle{fancy}
%----------------------------------------------------



% enter your details here----------------------------------

\lhead{\normalsize \textrm{Mysterious Stones}}
\chead{}
\rhead{\normalsize Bonan Zhao}
\lfoot{}
\cfoot{Page \thepage \hspace{1pt} of \pageref{LastPage}}
\rfoot{}
\setlength{\fboxrule}{4pt}\setlength{\fboxsep}{2ex}
\renewcommand{\headrulewidth}{0.4pt}
\renewcommand{\footrulewidth}{0.4pt}

	
\begin{document}


%----------------your title below -----------------------------

\begin{center}
{\bf Generative Model}

{\normalsize \today}
\end{center}


%---------------- start of document body------------------

Following definitions in previous notes \textit{Normative Model I},
refine a generative process to create causal relations (hypotheses) as follows.

\begin{enumerate}
\item Create one causal relation.
Start by sampling number of cause sentences (from a gamma distribution, say $\Gamma(1.5, 1)$?);
number of effect sentences is 2 by definition 
(one sentence for $R'$ lightness, and one sentence for $R'$ sidedness).

Each sentence is created by

\begin{enumerate}
	\item Sample relations, controlled by a relation parameter $\alpha$:

	$P(\text{pick }=) = P(\text{pick }\neq) = \alpha/2$, 
	$P(\text{pick }>) = P(\text{pick }<) = (1-\alpha)/2$.

	\item Sample a subject (left-hand side of the picked relation):

	For cause sentences, sample $A$ or $R$ are equally likely, 
	$P(\text{pick }A) = P(\text{pick }R) = 1/2$;
	effect sentences always take $R'$ as subjects.

	\item Sample an object (right-hand side of the picked relation):

	Objects can be absolute - an exact lightness or sidedness value,
	or relative - $A$'s lightness (sidedness) or $R$'s.
	Relative values, in addition, can combine with the increase or decrease by level 1 option (eg. $L(R')=L(A)+1)$.

	Assume that each type of values are picked equally like:

	$P(\text{pick }l_i) = \frac{1}{3|L|}$ 
	where $l_i$ is an exact lightness value, 
	and $|L|$ is the total number of lightness values (in current experiment setup);

	$P(\text{pick }s_i) = \frac{1}{3|S|}$
	where $s_i$ is an exact sidedness value
	and $|S|$ is the total number of sidedness values (in current experiment setup);

	$P(\text{pick }A) = P(\text{pick }R) = \frac{1}{6}$ for effect sentences,
	$P(\text{pick }A) = P(\text{pick }R) = \frac{1}{3}$ for cause sentences 
	(because for a cause sentence the object has to be different from the subject;

	$P(\text{pick }A, +1) = P(\text{pick }A, -1) = P(\text{pick }R, +1) = P(\text{pick }R, -1) = \frac{1}{3 \times 4} = \frac{1}{12}$ for effect sentences;
	and $\frac{1}{6}$ for cause sentences by similar reasons.


\end{enumerate}



\item With probability $\beta$ sample an extra causal relation
that will join the others as an \textit{if-else}.
$\beta$ decreases as the number of existing causal relations increases.

\end{enumerate} 

The above procedure generates \textit{one} hypothesis, 
that can potentially contain multiple causal relations by an \textit{if-else} relation.
















% ----------------end of document body---------------------



\end{document}